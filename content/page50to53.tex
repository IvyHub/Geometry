\documentclass[12pt]{article}

% Use this form to include EPS (latex) or PDF (pdflatex) files:
%\usepackage{asymptote}

% Use this form with latex or pdflatex to include inline LaTeX code by default:
\usepackage[inline]{asymptote}

\usepackage{fourier}
% Use this form with latex or pdflatex to create PDF attachments by default:
%\usepackage[attach]{asymptote}

% Enable this line to support the attach option:
%\usepackage[dvips]{attachfile2}

\begin{document}

% Optional subdirectory for latex files (no spaces):
\def\asylatexdir{}
% Optional subdirectory for asy files (no spaces):
\def\asydir{}

\begin{asydef}
// Global Asymptote definitions can be put here.
import three;
usepackage("bm");
texpreamble("\def\V#1{\bm{#1}}");
// One can globally override the default toolbar settings here:
// settings.toolbar=true;
\end{asydef}



\section*{Exercises}

\begin{enumerate}

\item Two circles intersect at $A$ and $B$, $AC$ and $AD$ are the diameters of two circles respectively, prove: $C, B$, and  $D$ are collinear.
 
\item Two circles  intersect at $A$ and $B$, $AD$ and $BF$ are the chords of the two circles, and   intersect with the other circles at $C$ and $E$ respectively, prove: $CF//DE$.

\item Two circles are tangent at $P$, one chord $AB$ of the first circle tangents the second circle at $C$. Assume the extension of $AP$ intersects the second circle at $D$, prove: $\angle BPC=\angle CPD$.

\item Given a half circle $O$, $AB$ is its diameter, $C$ is a point on this half circle, $CD\bot AB$ at $D$, $\odot P$ tangents $\odot O$ externally at $E$, tangents the line $CD$ at $F$, and $A, E$ are in the
same side of $CD$, prove: $A, E, F$ are collinear.

\end{enumerate} 

\section*{Exercises set 1}

 \begin{enumerate}
 \item Assume $AD$ is the median of $\triangle ABC$, $AE$ is the median of $\triangle ABD$, and $BA=BD$, prove: $AC=2AE$.
 %probem 1 graph
 
  \begin{asy}
    /* Geogebra to Asymptote conversion, documentation at artofproblemsolving.com/Wiki go to User:Azjps/geogebra */
import graph; size(4.cm); 
real labelscalefactor = 0.5; /* changes label-to-point distance */
pen dps = linewidth(0.7) + fontsize(10); defaultpen(dps); /* default pen style */ 
pen dotstyle = black; /* point style */ 
real xmin = -3.99399, xmax = 7.883700224892093, ymin = -8.713929985012639, ymax = 7.253134080690438;  /* image dimensions */

 /* draw figures */
draw((0.88,0.64)--(-1.8,-1.62), linewidth(2.)); 
draw((-1.8,-1.62)--(4.74,-1.58), linewidth(2.)); 
draw((4.74,-1.58)--(0.88,0.64), linewidth(2.)); 
draw((1.47,-1.6)--(0.88,0.64), linewidth(2.)); 
draw((-0.165,-1.61)--(0.88,0.64), linewidth(2.)); 
 /* dots and labels */
dot((0.88,0.64),dotstyle); 
label("$A$", (0.9584725332162531,0.8416488960065766), NE * labelscalefactor); 
dot((-1.8,-1.62),dotstyle); 
label("$B$", (-1.8773773820694024,-2.1585973763134354), NE * labelscalefactor); 
dot((4.74,-1.58),dotstyle); 
label("$C$", (4.513559745856966,-2.0147499522980925), NE * labelscalefactor); 
dot((1.47,-1.6),linewidth(4.pt) + dotstyle); 
label("$D$", (1.513312734033012,-2.055849216302476), NE * labelscalefactor); 
dot((-0.165,-1.61),dotstyle); 
label("$E$", (-0.27450569082098836,-2.1380477443112436), NE * labelscalefactor); 
clip((xmin,ymin)--(xmin,ymax)--(xmax,ymax)--(xmax,ymin)--cycle); 
 /* re-scale y/x */
currentpicture = yscale(1.115830390861128) * currentpicture; 
 /* end of picture */
 \end{asy} 
 
 
 \item Prove the perimeter of a triangle is greater than the sum of the three medians.
 
 \item Given a triangle $ABC$, assume $P$ is a point on the exterior angle bisector of $A$, prove: $PB+PC > AB+AC$.
% problem 3 , graph

 \begin{asy}
  /* Geogebra to Asymptote conversion, documentation at artofproblemsolving.com/Wiki go to User:Azjps/geogebra */
import graph; size(6.cm); 
real labelscalefactor = 0.5; /* changes label-to-point distance */
pen dps = linewidth(0.7) + fontsize(10); defaultpen(dps); /* default pen style */ 
pen dotstyle = black; /* point style */ 
real xmin = -12.65228500381642, xmax = 13.786436538120924, ymin = -10.810725412964283, ymax = 10.38936399682446;  /* image dimensions */

 /* draw figures */
draw((-0.6198018252876607,5.1780164070308325)--(-5.340027652692367,0.048522444302025035), linewidth(2.)); 
draw((-5.340027652692367,0.048522444302025035)--(2.436066918465675,0.0758069866569655), linewidth(2.)); 
draw((2.436066918465675,0.0758069866569655)--(-1.730923879399975,3.970554290423232), linewidth(2.)); 
draw((-1.730923879399975,3.970554290423232)--(2.46,4.13), linewidth(2.)); 
draw((-5.340027652692367,0.048522444302025035)--(2.46,4.13), linewidth(2.)); 
draw((2.46,4.13)--(2.436066918465675,0.0758069866569655), linewidth(2.)); 
 /* dots and labels */
dot((-0.6198018252876607,5.1780164070308325),dotstyle); 
label("$D$", (-0.5106636558678973,5.450861830580236), NE * labelscalefactor); 
dot((-5.340027652692367,0.048522444302025035),dotstyle); 
label("$B$", (-5.831149415081295,-0.49716840279678587), NE * labelscalefactor); 
dot((2.436066918465675,0.0758069866569655),dotstyle); 
label("$C$", (2.245075121981093,-0.5790220298616072), NE * labelscalefactor); 
dot((-1.730923879399975,3.970554290423232),dotstyle); 
label("$A$", (-2.2841589089390295,4.304911051672736), NE * labelscalefactor); 
dot((2.46,4.13),dotstyle); 
label("$P$", (2.572489630240379,4.414049221092498), NE * labelscalefactor); 
clip((xmin,ymin)--(xmin,ymax)--(xmax,ymax)--(xmax,ymin)--cycle); 
 /* re-scale y/x */
currentpicture = yscale(0.9976833976833988) * currentpicture; 
 /* end of picture */
 \end{asy}
 
 \item For a right triangle $ABC$, assume $AB$ is the hypotenuse, the perpendicular bisector $ME$ of $AB$ intersects the angle bisector of $C$ at $E$. Prove: $MC=ME$.
 %problem 4 graph
 
 \begin{asy}
  /* Geogebra to Asymptote conversion, documentation at artofproblemsolving.com/Wiki go to User:Azjps/geogebra */
import graph; size(6.cm); 
real labelscalefactor = 0.5; /* changes label-to-point distance */
pen dps = linewidth(0.7) + fontsize(10); defaultpen(dps); /* default pen style */ 
pen dotstyle = black; /* point style */ 
real xmin = -4.232814273524748, xmax = 7.719872903505422, ymin = -4.916564176694168, ymax = 3.8202809741350294;  /* image dimensions */
pen qqwuqq = rgb(0.,0.39215686274509803,0.); 

draw((2.958519252677728,2.218889439508296)--(3.139629813169432,1.9774086921860237)--(3.381110560491704,2.1585192526777277)--(3.2,2.4)--cycle, linewidth(2.) + qqwuqq); 
 /* draw figures */
draw((0.,0.)--(5.,0.), linewidth(2.)); 
draw((0.,0.)--(3.2,2.4), linewidth(2.)); 
draw((3.2,2.4)--(5.,0.), linewidth(2.)); 
draw((2.5,-2.5)--(3.2,2.4), linewidth(2.)); 
draw((2.5,-2.5)--(2.5,0.), linewidth(2.)); 
draw((2.329554410482433,0.)--(2.3258458953944197,-0.1928921158747807), linewidth(2.)); 
draw((2.5,-0.1928921158747805)--(2.3258458953944197,-0.1928921158747807), linewidth(2.)); 
 /* dots and labels */
dot((0.,0.),dotstyle); 
label("$A$", (-0.3197321619970134,-0.34893014832906627), NE * labelscalefactor); 
dot((5.,0.),dotstyle); 
label("$B$", (5.044747678170099,-0.3062419798396728), NE * labelscalefactor); 
dot((3.2,2.4),dotstyle); 
label("$C$", (3.20915643312618,2.596553477439083), NE * labelscalefactor); 
dot((2.5,-2.5),dotstyle); 
label("$E$", (2.4407694003170977,-2.8248439207138873), NE * labelscalefactor); 
dot((2.5,0.),linewidth(2.pt) + dotstyle); 
label("$M$", (2.0992640524019497,0.1348690945507264), NE * labelscalefactor); 
dot((2.329554410482433,0.),linewidth(1.pt) + dotstyle); 
dot((2.3258458953944197,-0.1928921158747807),linewidth(1.pt) + dotstyle); 
dot((2.5,-0.1928921158747805),linewidth(1.pt) + dotstyle); 
clip((xmin,ymin)--(xmin,ymax)--(xmax,ymax)--(xmax,ymin)--cycle); 
 /* re-scale y/x */
currentpicture = yscale(0.9804560260586318) * currentpicture; 
 /* end of picture */
 \end{asy}
 \item For an isosceles triangle $ABC$ with $AB=AC$, $CX$ is the altitude on $AB$, $XP\bot BC$ at $P$. Prove: $AB^2=PA^2+PX^2$.
 %problem 5, graph
 
 \begin{asy}
  /* Geogebra to Asymptote conversion, documentation at artofproblemsolving.com/Wiki go to User:Azjps/geogebra */
import graph; size(6.cm); 
real labelscalefactor = 0.5; /* changes label-to-point distance */
pen dps = linewidth(0.7) + fontsize(10); defaultpen(dps); /* default pen style */ 
pen dotstyle = black; /* point style */ 
real xmin = -4.781485122045432, xmax = 4.715976662930288, ymin = -2.2036406131540254, ymax = 4.410663129954066;  /* image dimensions */

 /* draw figures */
draw((0.,4.)--(-2.,0.), linewidth(2.)); 
draw((-2.,0.)--(2.,0.), linewidth(2.)); 
draw((0.,4.)--(2.,0.), linewidth(2.)); 
draw((2.,0.)--(-1.2,1.6), linewidth(2.)); 
draw((-1.1587380349051477,1.6825239301897046)--(-1.072492062008145,1.637421945759577), linewidth(2.)); 
draw((-1.1177120230662512,1.5588560115331256)--(-1.072492062008145,1.637421945759577), linewidth(2.)); 
draw((-1.2,1.6)--(-1.2,0.), linewidth(2.)); 
draw((-1.2,0.16606273511580258)--(-1.0324697551333837,0.16331012749915502), linewidth(2.)); 
draw((-1.0297171475167362,0.)--(-1.0324697551333837,0.16331012749915502), linewidth(2.)); 
 /* dots and labels */
dot((0.,4.),dotstyle); 
label("$A$", (0.06900428956716685,4.1280005768297885), NE * labelscalefactor); 
dot((-2.,0.),dotstyle); 
label("$B$", (-2.2375221439269364,-0.3041482561588813), NE * labelscalefactor); 
dot((2.,0.),dotstyle); 
label("$C$", (2.149400680561848,0.05765981184019375), NE * labelscalefactor); 
dot((-1.2,1.6),linewidth(2.pt) + dotstyle); 
label("$X$", (-1.5139060079287863,1.6518766114611183), NE * labelscalefactor); 
dot((-1.2,0.),linewidth(2.pt) + dotstyle); 
label("$P$", (-1.231243454804509,-0.270228749783968), NE * labelscalefactor); 
dot((-1.1587380349051477,1.6825239301897046),linewidth(1.pt) + dotstyle); 
dot((-1.072492062008145,1.637421945759577),linewidth(1.pt) + dotstyle); 
dot((-1.1177120230662512,1.5588560115331256),linewidth(1.pt) + dotstyle); 
dot((-1.2,0.16606273511580258),linewidth(1.pt) + dotstyle); 
dot((-1.0324697551333837,0.16331012749915502),linewidth(1.pt) + dotstyle); 
dot((-1.0297171475167362,0.),linewidth(1.pt) + dotstyle); 
clip((xmin,ymin)--(xmin,ymax)--(xmax,ymax)--(xmax,ymin)--cycle); 
 /* re-scale y/x */
currentpicture = yscale(1.0051282051282049) * currentpicture; 
 /* end of picture */
 \end{asy}
 
 \item Prove the length of the diagonal of a rectangle is longer than any line segment between the two opposite sides.
 
 \item For the square $ABCD$, assume $E$ is the midpoint of $CD$, $BF\bot AE$ at $F$, prove: $CF=CB$.
 %problem 7, graph is here
 
 \begin{asy}
  /* Geogebra to Asymptote conversion, documentation at artofproblemsolving.com/Wiki go to User:Azjps/geogebra */
import graph; size(5.cm); 
real labelscalefactor = 0.5; /* changes label-to-point distance */
pen dps = linewidth(0.7) + fontsize(10); defaultpen(dps); /* default pen style */ 
pen dotstyle = black; /* point style */ 
real xmin = -7.086270484572427, xmax = 17.616372676863413, ymin = -8.648587670983188, ymax = 8.555038816445293;  /* image dimensions */

 /* draw figures */
draw((0.,0.)--(4.,0.), linewidth(2.)); 
draw((4.,0.)--(4.,4.), linewidth(2.)); 
draw((4.,4.)--(0.,4.), linewidth(2.)); 
draw((0.,4.)--(0.,0.), linewidth(2.)); 
draw((0.,0.)--(2.,4.), linewidth(2.)); 
draw((4.,0.)--(0.796,1.592), linewidth(2.)); 
draw((0.796,1.592)--(4.,4.), linewidth(2.)); 
draw((0.6695301316495708,1.3390602632991415)--(0.9053047501922775,1.2285488382042102), linewidth(2.)); 
draw((0.9053047501922775,1.2285488382042102)--(1.032099380779174,1.4746871990635317), linewidth(2.)); 
 /* dots and labels */
dot((0.,0.),linewidth(4.pt) + dotstyle); 
label("$A$", (-0.44008315780516594,-0.5025970094144881), NE * labelscalefactor); 
dot((4.,0.),dotstyle); 
label("$B$", (4.118142663650257,-0.5320049179400068), NE * labelscalefactor); 
dot((4.,4.),dotstyle); 
label("$C$", (3.971103121022663,4.290892080245071), NE * labelscalefactor); 
dot((0.,4.),dotstyle); 
label("$D$", (-0.14600407254997738,4.261484171719553), NE * labelscalefactor); 
dot((2.,4.),dotstyle); 
label("$E$", (1.9419574327618614,4.290892080245071), NE * labelscalefactor); 
dot((0.796,1.592),linewidth(2.pt) + dotstyle); 
label("$F$", (0.2657066468072866,1.4677328617952696), NE * labelscalefactor); 
dot((0.6695301316495708,1.3390602632991415),linewidth(1.pt) + dotstyle); 
dot((0.9053047501922775,1.2285488382042102),linewidth(1.pt) + dotstyle); 
dot((1.032099380779174,1.4746871990635317),linewidth(1.pt) + dotstyle); 
clip((xmin,ymin)--(xmin,ymax)--(xmax,ymax)--(xmax,ymin)--cycle); 
 /* re-scale y/x */
currentpicture = yscale(0.9993846153846184) * currentpicture; 
 /* end of picture */
   \end{asy}
 \item For an isosceles triangle $ABC, AB=AC$, the circle which takes $AB$ as a diameter intersects $AC$ and $BC$ at $E$ and $D$ respectively. Make $DF\bot AC$ at $F$, prove: $DF^2=EF\cdot FA$.
 
 %graphi is here, problem 8
 \vskip 0.1 cm
  
 \begin{asy}
  /* Geogebra to Asymptote conversion, documentation at artofproblemsolving.com/Wiki go to User:Azjps/geogebra */
import graph; size(6.cm); 
real labelscalefactor = 0.5; /* changes label-to-point distance */
pen dps = linewidth(0.7) + fontsize(10); defaultpen(dps); /* default pen style */ 
pen dotstyle = black; /* point style */ 
real xmin = -4.2980315390440085, xmax = 3.7288327904638505, ymin = -3.633267873088471, ymax = 2.638769422039532;  /* image dimensions */

 /* draw figures */
draw(shift((0.,0.)) * scale(2., 2.)*unitcircle, linewidth(2.)); 
draw((-1.01,-1.73)--(1.01,1.73), linewidth(2.)); 
draw((-1.01,-1.73)--(3.03,-1.73), linewidth(2.)); 
draw((3.03,-1.73)--(1.01,1.73), linewidth(2.)); 
draw((1.01,-1.73)--(2.52,-0.85), linewidth(2.)); 
draw((2.251733624816546,-1.0063406689810859)--(2.410613439278983,-1.2990960283152644), linewidth(2.)); 
draw((2.675846030858552,-1.1233798350349462)--(2.410613439278983,-1.2990960283152644), linewidth(2.)); 
 /* dots and labels */
dot((-1.01,-1.73),dotstyle); 
label("$B$", (-1.2165888518793806,-2.0463668707066867), NE * labelscalefactor); 
dot((1.01,1.73),dotstyle); 
label("$A$", (1.0588088816944998,1.8411207700486885), NE * labelscalefactor); 
dot((3.03,-1.73),dotstyle); 
label("$C$", (3.1075064720709826,-1.8700445371087109), NE * labelscalefactor); 
dot((1.01,-1.73),linewidth(4.pt) + dotstyle); 
label("$D$", (0.9580532624956565,-2.0295742675068795), NE * labelscalefactor); 
dot((2.,0.03),linewidth(4.pt) + dotstyle); 
label("$E$", (2.0495724704831266,0.027519624469506898), NE * labelscalefactor); 
dot((2.52,-0.85),linewidth(2.pt) + dotstyle); 
label("$F$", (2.5533505664773437,-0.8205068371207586), NE * labelscalefactor); 
dot((2.821496270204694,-1.3728599479743768),linewidth(1.pt) + dotstyle); 
dot((2.251733624816546,-1.0063406689810859),linewidth(1.pt) + dotstyle); 
dot((2.410613439278983,-1.2990960283152644),linewidth(1.pt) + dotstyle); 
dot((2.675846030858552,-1.1233798350349462),linewidth(1.pt) + dotstyle); 
clip((xmin,ymin)--(xmin,ymax)--(xmax,ymax)--(xmax,ymin)--cycle); 
 /* re-scale y/x */
currentpicture = yscale(0.9598393574297188) * currentpicture; 
 /* end of picture */
 \end{asy}
 \item Prove: For a triangle, the symmetric points of the orthocenter along the sides are on the circumcircle.
 
 \item In the following figure, $AB$ is a diameter of $\odot O$, $AT$ is a tangent line of $\odot O$, $P$ is on the extension of $BM$ such that $PT\bot AT, PT=PM$. Prove: $PB=AB$.
 
 %graph here problem 10
 
 \begin{asy}
   /* Geogebra to Asymptote conversion, documentation at artofproblemsolving.com/Wiki go to User:Azjps/geogebra */
import graph; size(6.cm); 
real labelscalefactor = 0.5; /* changes label-to-point distance */
pen dps = linewidth(0.7) + fontsize(10); defaultpen(dps); /* default pen style */ 
pen dotstyle = black; /* point style */ 
real xmin = -4.473104296154636, xmax = 6.243997739846385, ymin = -4.581187822491017, ymax = 4.297987097254904;  /* image dimensions */
pen qqwuqq = rgb(0.,0.39215686274509803,0.); 

draw((-2.,2.336565393128929)--(-1.7365653931289289,2.336565393128929)--(-1.7365653931289289,2.6)--(-2.,2.6)--cycle, linewidth(2.) + qqwuqq); 
 /* draw figures */
draw(shift((0.,0.)) * scale(2., 2.)*unitcircle, linewidth(2.)); 
draw((-2.,0.)--(2.,0.), linewidth(2.)); 
draw((-2.,0.)--(-2.,2.6), linewidth(2.)); 
draw((-2.,2.6)--(-1.03,2.6), linewidth(2.)); 
draw((-1.03,2.6)--(2.,0.), linewidth(2.)); 
 /* dots and labels */
dot((-2.,0.),linewidth(4.pt) + dotstyle); 
label("$A$", (-1.9521637245344654,0.10055895337501349), NE * labelscalefactor); 
dot((2.,0.),linewidth(4.pt) + dotstyle); 
label("$B$", (2.046569595966495,0.10055895337501349), NE * labelscalefactor); 
dot((-2.,2.6),dotstyle); 
label("$T$", (-1.9521637245344654,2.7208469366846493), NE * labelscalefactor); 
dot((-1.03,2.6),dotstyle); 
label("$P$", (-0.9835264605621832,2.7208469366846493), NE * labelscalefactor); 
dot((0.,0.),dotstyle); 
label("$O$", (0.047202935716014735,0.1253958062973797), NE * labelscalefactor); 
dot((-0.3037344190102191,1.976801811691937),linewidth(4.pt) + dotstyle); 
label("$M$", (-0.2508392993523798,2.0750887607031276), NE * labelscalefactor); 
clip((xmin,ymin)--(xmin,ymax)--(xmax,ymax)--(xmax,ymin)--cycle); 
 /* re-scale y/x */
currentpicture = yscale(0.9655944055944056) * currentpicture; 
 /* end of picture */
 \end{asy}
 \item In the following figure, $AB$ is a diameter of $\odot O$, $P$ is a point on the circle, $Q$ is the midpoint of the arc $\widearc{BP}$, the tangent line $QH$ intersects $AP$ at $H$, prove: $QH\bot AP$.
 %problem 11 graph
 
 \begin{asy}
  /* Geogebra to Asymptote conversion, documentation at artofproblemsolving.com/Wiki go to User:Azjps/geogebra */
import graph; size(5.2cm); 
real labelscalefactor = 0.5; /* changes label-to-point distance */
pen dps = linewidth(0.7) + fontsize(10); defaultpen(dps); /* default pen style */ 
pen dotstyle = black; /* point style */ 
real xmin = -4.4731042961546335, xmax = 5.958373931239167, ymin = -2.966792382537209, ymax = 4.297987097254902;  /* image dimensions */

 /* draw figures */
draw(shift((0.,0.)) * scale(2., 2.)*unitcircle, linewidth(2.)); 
draw((-2.,0.)--(2.,0.), linewidth(2.)); 
draw((-2.,0.)--(0.9537480673823792,2.3110388634656065), linewidth(2.)); 
draw((1.58,1.23)--(0.9537480673823792,2.3110388634656065), linewidth(2.)); 
 /* dots and labels */
dot((-2.,0.),linewidth(4.pt) + dotstyle); 
label("$A$", (-2.349553371292324,-0.09813587000391401), NE * labelscalefactor); 
dot((2.,0.),linewidth(4.pt) + dotstyle); 
label("$B$", (2.1583354341171392,-0.18506485523219568), NE * labelscalefactor); 
dot((0.,0.),dotstyle); 
label("$O$", (0.04720293571601328,0.1253958062973817), NE * labelscalefactor); 
dot((0.9537480673823792,2.3110388634656065),dotstyle); 
label("$H$", (1.0034217732271116,2.4352231280774372), NE * labelscalefactor); 
dot((1.58,1.23),dotstyle); 
label("$Q$", (1.6243430962862664,1.354820025954508), NE * labelscalefactor); 
dot((0.4811375657135137,1.9412641867760487),linewidth(4.pt) + dotstyle); 
label("$P$", (0.35766359724559066,2.0626703342419446), NE * labelscalefactor); 
clip((xmin,ymin)--(xmin,ymax)--(xmax,ymax)--(xmax,ymin)--cycle); 
 /* re-scale y/x */
currentpicture = yscale(0.9664694280078895) * currentpicture; 
 /* end of picture */
 \end{asy}
 
\item In the following figure, two circles  tangent internally at $P$, a secant intersects the two circles at $A, B, C, D$, prove: $\angle APB=\angle CPD$.
%problem 12, graph

\begin{asy}
 /* Geogebra to Asymptote conversion, documentation at artofproblemsolving.com/Wiki go to User:Azjps/geogebra */
import graph; size(6.cm); 
real labelscalefactor = 0.5; /* changes label-to-point distance */
pen dps = linewidth(0.7) + fontsize(10); defaultpen(dps); /* default pen style */ 
pen dotstyle = black; /* point style */ 
real xmin = -4.4731042961546335, xmax = 5.958373931239167, ymin = -2.966792382537209, ymax = 4.297987097254902;  /* image dimensions */

 /* draw figures */
draw(shift((0.,0.)) * scale(2., 2.)*unitcircle, linewidth(2.)); 
draw(shift((-0.7599947842608878,-0.358922825688759)) * scale(1.144113882562022, 1.144113882562022)*unitcircle, linewidth(2.)); 
draw((-1.9670045595430818,0.361791463051198)--(1.974168812869036,0.3204020884689435), linewidth(2.)); 
draw((-1.795492987832248,-0.8454822225475906)--(-1.9670045595430818,0.361791463051198), linewidth(2.)); 
draw((-1.795492987832248,-0.8454822225475906)--(-1.6512502196468668,0.3584754773243538), linewidth(2.)); 
draw((-1.795492987832248,-0.8454822225475906)--(0.14613035480935482,0.3395997642223174), linewidth(2.)); 
draw((-1.795492987832248,-0.8454822225475906)--(1.974168812869036,0.3204020884689435), linewidth(2.)); 
 /* dots and labels */
dot((-0.7599947842608878,-0.358922825688759),dotstyle); 
dot((-1.9670045595430818,0.361791463051198),dotstyle); 
label("$A$", (-2.250205959602859,0.26199849737039577), NE * labelscalefactor); 
dot((1.974168812869036,0.3204020884689435),dotstyle); 
label("$D$", (2.0217327430441254,0.44827489428814216), NE * labelscalefactor); 
dot((-1.795492987832248,-0.8454822225475906),dotstyle); 
label("$P$", (-2.0515111362239296,-1.1785389721268433), NE * labelscalefactor); 
dot((-1.6512502196468668,0.3584754773243538),linewidth(4.pt) + dotstyle); 
label("$B$", (-1.5423556513154228,0.1378142327585648), NE * labelscalefactor); 
dot((0.14613035480935482,0.3395997642223174),linewidth(4.pt) + dotstyle); 
label("$C$", (0.19622405325021042,0.4358564678269591), NE * labelscalefactor); 
clip((xmin,ymin)--(xmin,ymax)--(xmax,ymax)--(xmax,ymin)--cycle); 
 /* re-scale y/x */
currentpicture = yscale(1.0290598290598292) * currentpicture; 
 /* end of picture */
\end{asy}
\item In the following figure, two circles  tangent externally at $P$, a secant intersects the two circles at $A, B, C, D$, prove: $\angle APD+\angle BPC=180^\circ$.
%problem 13 , graph

\begin{asy}
 /* Geogebra to Asymptote conversion, documentation at artofproblemsolving.com/Wiki go to User:Azjps/geogebra */
import graph; size(5.1cm); 
real labelscalefactor = 0.5; /* changes label-to-point distance */
pen dps = linewidth(0.7) + fontsize(10); defaultpen(dps); /* default pen style */ 
pen dotstyle = black; /* point style */ 
real xmin = -4.4731042961546335, xmax = 2.891022595326942, ymin = -2.966792382537209, ymax = 4.297987097254902;  /* image dimensions */

 /* draw figures */
draw(shift((0.,0.)) * scale(1.4142135623730951, 1.4142135623730951)*unitcircle, linewidth(2.)); 
draw(shift((-2.3619717977535073,0.)) * scale(0.9219393033124031, 0.9219393033124031)*unitcircle, linewidth(2.)); 
draw((-3.136118898030146,0.5006679000349743)--(1.,1.), linewidth(2.)); 
draw((-1.4404026430075607,0.026122251300388356)--(-3.136118898030146,0.5006679000349743), linewidth(2.)); 
draw((-1.4404026430075607,0.026122251300388356)--(1.,1.), linewidth(2.)); 
draw((-1.4404026430075607,0.026122251300388356)--(-1.7471093919730598,0.6869616444533417), linewidth(2.)); 
draw((-1.4404026430075607,0.026122251300388356)--(-1.209250913144812,0.7332886396627447), linewidth(2.)); 
 /* dots and labels */
dot((-3.136118898030146,0.5006679000349743),dotstyle); 
label("$A$", (-3.3430274881869715,0.7214802764341702), NE * labelscalefactor); 
dot((1.,1.),dotstyle); 
label("$D$", (1.0530954790718439,1.1188699231920294), NE * labelscalefactor); 
dot((-1.4404026430075607,0.026122251300388356),dotstyle); 
label("$P$", (-1.318823975014127,-0.22232013461574496), NE * labelscalefactor); 
dot((-1.7471093919730598,0.6869616444533417),dotstyle); 
label("$B$", (-1.69137676884962,0.808409261662452), NE * labelscalefactor); 
dot((-1.209250913144812,0.7332886396627447),linewidth(4.pt) + dotstyle); 
label("$C$", (-1.306405548552944,0.8580829675071843), NE * labelscalefactor); 
clip((xmin,ymin)--(xmin,ymax)--(xmax,ymax)--(xmax,ymin)--cycle); 
 /* re-scale y/x */
currentpicture = yscale(0.993799229093347) * currentpicture; 
 /* end of picture */
\end{asy}
\item Two circles intersect at $A,B$, the line passes $A$ intersect the two circles at $C$ and $D$, the tangent lines at $C$ and $D$ intersect at $P$. Prove: $B, C, P, D$ are cyclic.
%problem 14, graph

\begin{asy}
 /* Geogebra to Asymptote conversion, documentation at artofproblemsolving.com/Wiki go to User:Azjps/geogebra */
import graph; size(6.cm); 
real labelscalefactor = 0.5; /* changes label-to-point distance */
pen dps = linewidth(0.7) + fontsize(10); defaultpen(dps); /* default pen style */ 
pen dotstyle = black; /* point style */ 
real xmin = -4.435849016771084, xmax = 2.928277874710491, ymin = -2.966792382537209, ymax = 4.297987097254902;  /* image dimensions */

 /* draw figures */
draw(shift((0.,0.)) * scale(1.4142135623730951, 1.4142135623730951)*unitcircle, linewidth(2.)); 
draw(shift((-2.,0.)) * scale(1., 1.)*unitcircle, linewidth(2.)); 
draw((-2.8551526319614684,0.5183762880855693)--(1.118100142291255,0.8659399931913733), linewidth(2.)); 
draw((-2.8551526319614684,0.5183762880855693)--(-0.99,3.59), linewidth(2.)); 
draw((-0.99,3.59)--(1.118100142291255,0.8659399931913733), linewidth(2.)); 
 /* dots and labels */
dot((-2.8551526319614684,0.5183762880855693),dotstyle); 
label("$C$", (-3.119495811885676,0.6221328647447055), NE * labelscalefactor); 
dot((1.118100142291255,0.8659399931913733),dotstyle); 
label("$D$", (1.1648613172224918,0.9946856585801983), NE * labelscalefactor); 
dot((-0.99,3.59),dotstyle); 
label("$P$", (-0.9462711811786343,3.714321053579296), NE * labelscalefactor); 
dot((-1.25,0.6614378277661476),linewidth(4.pt) + dotstyle); 
label("$A$", (-1.3809161073200427,0.8705013939683673), NE * labelscalefactor); 
dot((-1.25,-0.6614378277661477),linewidth(4.pt) + dotstyle); 
label("$B$", (-1.1201291516351977,-0.7687308989078011), NE * labelscalefactor); 
clip((xmin,ymin)--(xmin,ymax)--(xmax,ymax)--(xmax,ymin)--cycle); 
 /* re-scale y/x */
currentpicture = yscale(0.9967806267806267) * currentpicture; 
 /* end of picture */
\end{asy}

\item In $\triangle ABC, \angle C=90^\circ, CD $ is the altitude, the circle which takes $CD$ as a diameter intersects $AC$ and $BC$ at $E$ and $F$ respectively, prove: $\frac{BF}{AE}=\frac{BC^3}{AC^3}$.
%problem 15, graph

\begin{asy}
 /* Geogebra to Asymptote conversion, documentation at artofproblemsolving.com/Wiki go to User:Azjps/geogebra */
import graph; size(6.cm); 
real labelscalefactor = 0.5; /* changes label-to-point distance */
pen dps = linewidth(0.7) + fontsize(10); defaultpen(dps); /* default pen style */ 
pen dotstyle = black; /* point style */ 
real xmin = -4.50087, xmax = 5.682763627804912, ymin = -4.998866289315185, ymax = 3.260823526070558;  /* image dimensions */

 /* draw figures */
draw(shift((0.,0.)) * scale(1.4142135623730951, 1.4142135623730951)*unitcircle, linewidth(2.)); 
draw((-2.956943953062594,-1.41264590586317)--(2.69,-1.41), linewidth(2.)); 
draw((-2.956943953062594,-1.41264590586317)--(0.,1.41), linewidth(2.)); 
draw((0.,1.41)--(2.69,-1.41), linewidth(2.)); 
draw((0.,1.41)--(0.,-1.41), linewidth(2.)); 
draw((0.,-1.0261070393299605)--(0.3921201212819644,-1.0297514534632959), linewidth(2.)); 
draw((0.40000518830852183,-1.4110729893459624)--(0.3921201212819644,-1.0297514534632959), linewidth(2.)); 
 /* dots and labels */
dot((0.,-1.41),dotstyle); 
label("$D$", (-0.21069644127219764,-1.8033504169638062), NE * labelscalefactor); 
dot((0.,1.41),dotstyle); 
label("$C$", (-0.15541069953601463,1.6354227190267776), NE * labelscalefactor); 
dot((-2.956943953062594,-1.41264590586317),dotstyle); 
label("$A$", (-3.2735265334567374,-1.7591218235748596), NE * labelscalefactor); 
dot((2.69,-1.41),dotstyle); 
label("$B$", (2.509362052148007,-1.7812361202693328), NE * labelscalefactor); 
dot((0.,-1.0261070393299605),linewidth(1.pt) + dotstyle); 
dot((0.3921201212819644,-1.0297514534632959),linewidth(1.pt) + dotstyle); 
dot((0.40000518830852183,-1.4110729893459624),linewidth(1.pt) + dotstyle); 
dot((-1.4128850371227102,0.06128516847294829),linewidth(4.pt) + dotstyle); 
label("$E$", (-1.7476400615380858,0.0984790987608897), NE * labelscalefactor); 
dot((1.4124450021631278,-0.07070442605948733),linewidth(4.pt) + dotstyle); 
label("$F$", (1.4589329591605298,0.02107906033023348), NE * labelscalefactor); 
clip((xmin,ymin)--(xmin,ymax)--(xmax,ymax)--(xmax,ymin)--cycle); 
 /* re-scale y/x */
currentpicture = yscale(0.94524765729585) * currentpicture; 
 /* end of picture */
 \end{asy}
\item In the following $\triangle ABC, \angle B=3\angle C, AD $ is the angle bisector of $\angle A$, $BD\bot AD$. Prove: $BD=\frac12(AC-AB)$.

%problem 16, graph

\begin{asy}
 /* Geogebra to Asymptote conversion, documentation at artofproblemsolving.com/Wiki go to User:Azjps/geogebra */
import graph; size(6.cm); 
real labelscalefactor = 0.5; /* changes label-to-point distance */
pen dps = linewidth(0.7) + fontsize(10); defaultpen(dps); /* default pen style */ 
pen dotstyle = black; /* point style */ 
real xmin = -1.4556955905720552, xmax = 3.7597272456823423, ymin = -2.2937230767348784, ymax = 0.8614189700167468;  /* image dimensions */
pen qqwuqq = rgb(0.,0.39215686274509803,0.); 

draw((0.,-0.3855885981094934)--(-0.11441140189050655,-0.3855885981094934)--(-0.11441140189050655,-0.5)--(0.,-0.5)--cycle, linewidth(2.) + qqwuqq); 
 /* draw figures */
draw((-1.,-0.5)--(0.,0.6940870071188128), linewidth(2.)); 
draw((0.,0.6940870071188128)--(1.,-0.5), linewidth(2.)); 
draw((2.29,-2.02)--(1.,-0.5), linewidth(2.)); 
draw((2.29,-2.02)--(-1.,-0.5), linewidth(2.)); 
draw((-1.,-0.5)--(0.,-0.5), linewidth(2.)); 
draw((0.,-0.5)--(0.,0.6940870071188128), linewidth(2.)); 
 /* dots and labels */
dot((-1.,-0.5),dotstyle); 
label("$B$", (-1.094337441627852,-0.7458157820037392), NE * labelscalefactor); 
dot((0.,0.6940870071188128),dotstyle); 
label("$A$", (0.05445786770222743,0.6942234085649512), NE * labelscalefactor); 
dot((2.29,-2.02),dotstyle); 
label("$C$", (2.3089012446974064,-1.9647253590394098), NE * labelscalefactor); 
dot((0.,-0.5),dotstyle); 
label("$D$", (0.08142489374283962,-0.5192927632625969), NE * labelscalefactor); 
clip((xmin,ymin)--(xmin,ymax)--(xmax,ymax)--(xmax,ymin)--cycle); 
 /* re-scale y/x */
currentpicture = yscale(0.9642450142450146) * currentpicture; 
 /* end of picture */
\end{asy}
\item In right triangle $ABC, \angle A=90^\circ, AD $ is the altitude on $BC, BF$ is the angle bisector of $\angle B$, $AD$ and $BF$ intersect at $E$, $EG//BC$. Prove: $CG=AF$.
%problem 17

\begin{asy}
 /* Geogebra to Asymptote conversion, documentation at artofproblemsolving.com/Wiki go to User:Azjps/geogebra */
import graph; size(6.8cm); 
real labelscalefactor = 0.5; /* changes label-to-point distance */
pen dps = linewidth(0.7) + fontsize(10); defaultpen(dps); /* default pen style */ 
pen dotstyle = black; /* point style */ 
real xmin = -2.0085778964173815, xmax = 1.4383431546019838, ymin = -1.284224087603806, ymax = 1.7480297693079643;  /* image dimensions */
pen qqwuqq = rgb(0.,0.39215686274509803,0.); 

draw((0.10995524433340398,0.)--(0.10995524433340399,0.10995524433340398)--(0.,0.10995524433340398)--(0.,0.)--cycle, linewidth(2.) + qqwuqq); 
 /* draw figures */
draw((0.,1.103313434547969)--(-1.7110913323403383,0.), linewidth(2.)); 
draw((-1.7110913323403383,0.)--(0.71,0.), linewidth(2.)); 
draw((0.,1.103313434547969)--(0.,0.), linewidth(2.)); 
draw((0.,1.103313434547969)--(0.71,0.), linewidth(2.)); 
draw((-1.7110913323403383,0.)--(0.328152103509231,0.5933773438762644), linewidth(2.)); 
draw((0.,0.5)--(0.38662981405451946,0.5025051696984838), linewidth(2.)); 
 /* dots and labels */
dot((0.,1.103313434547969),dotstyle); 
label("$A$", (-0.04409206433266045,1.16231235763099), NE * labelscalefactor); 
dot((-1.7110913323403383,0.),dotstyle); 
label("$B$", (-1.7753275997318603,-0.14388930380792658), NE * labelscalefactor); 
dot((0.71,0.),dotstyle); 
label("$C$", (0.7645089641771455,-0.04022250528102844), NE * labelscalefactor); 
dot((0.,0.),dotstyle); 
label("$D$", (-0.03372538447997063,-0.11278926424985714), NE * labelscalefactor); 
dot((0.,0.5),linewidth(4.pt) + dotstyle); 
label("$E$", (0.02329135470982338,0.5403115664696011), NE * labelscalefactor); 
dot((0.328152103509231,0.5933773438762644),dotstyle); 
label("$F$", (0.3498417700695527,0.6439783649964992), NE * labelscalefactor); 
dot((0.38662981405451946,0.5025051696984838),dotstyle); 
label("$G$", (0.40685850925934675,0.5558615862486358), NE * labelscalefactor); 
clip((xmin,ymin)--(xmin,ymax)--(xmax,ymax)--(xmax,ymin)--cycle); 
 /* re-scale y/x */
currentpicture = yscale(1.0030165912518862) * currentpicture; 
 /* end of picture */
\end{asy}
\item In the following figure, $D$ and $E$ are the midpoints of $AB$ and $AC$ respectively,  $AB>AC$. $F$ is a point between $B$ and $D$ such that $BF=AE$.  $AH$ is the angle bisector of 
$\angle BAC$, $FH\bot AH$, and $FH$ intersects $BC$ at $M$. Prove: $BM=MC$.
%problem 18, graph

\begin{asy}
 /* Geogebra to Asymptote conversion, documentation at artofproblemsolving.com/Wiki go to User:Azjps/geogebra */
import graph; size(6.cm); 
real labelscalefactor = 0.5; /* changes label-to-point distance */
pen dps = linewidth(0.7) + fontsize(10); defaultpen(dps); /* default pen style */ 
pen dotstyle = black; /* point style */ 
real xmin = -4.477677658842574, xmax = 0.2213479843923375, ymin = -2.2000230178843236, ymax = 1.4409438712447036;  /* image dimensions */

 /* draw figures */
draw((-1.,0.5)--(-3.556526183665778,-0.46167474859710544), linewidth(2.)); 
draw((-3.556526183665778,-0.46167474859710544)--(-0.30058754755411127,-0.4983333864943361), linewidth(2.)); 
draw((-0.30058754755411127,-0.4983333864943361)--(-1.,0.5), linewidth(2.)); 
draw((-1.,0.5)--(-1.39,-0.75), linewidth(2.)); 
draw((-2.99,-0.25)--(-1.39,-0.75), linewidth(2.)); 
draw((-1.4518662326208744,-0.7306668023059767)--(-1.4250210901899658,-0.6753361919983835), linewidth(2.)); 
draw((-1.4250210901899658,-0.6753361919983835)--(-1.3724793032013831,-0.6938439205172537), linewidth(2.)); 
 /* dots and labels */
dot((-1.,0.5),dotstyle); 
label("$A$", (-0.9736360202448849,0.5633774928392458), NE * labelscalefactor); 
dot((-3.556526183665778,-0.46167474859710544),dotstyle); 
label("$B$", (-3.8117230312582886,-0.6751736369670387), NE * labelscalefactor); 
dot((-0.30058754755411127,-0.4983333864943361),dotstyle); 
label("$C$", (-0.2641142674915341,-0.6440542618462778), NE * labelscalefactor); 
dot((-2.2782630918328888,0.019162625701447278),linewidth(4.pt) + dotstyle); 
label("$D$", (-2.305545275413456,0.07791524095537547), NE * labelscalefactor); 
dot((-0.6502937737770557,8.333067528319416e-4),linewidth(4.pt) + dotstyle); 
label("$E$", (-0.6250990188923617,0.053019740858766734), NE * labelscalefactor); 
dot((-2.99,-0.25),linewidth(4.pt) + dotstyle); 
label("$F$", (-2.99639540309435,-0.1959352601073206), NE * labelscalefactor); 
dot((-1.39,-0.75),linewidth(2.pt) + dotstyle); 
label("$H$", (-1.3657401467664736,-0.8120988874983868), NE * labelscalefactor); 
dot((-1.4518662326208744,-0.7306668023059767),linewidth(1.pt) + dotstyle); 
dot((-1.4250210901899658,-0.6753361919983835),linewidth(1.pt) + dotstyle); 
dot((-1.3724793032013831,-0.6938439205172537),linewidth(1.pt) + dotstyle); 
clip((xmin,ymin)--(xmin,ymax)--(xmax,ymax)--(xmax,ymin)--cycle); 
 /* re-scale y/x */
currentpicture = yscale(1.010968660968664) * currentpicture; 
 /* end of picture */
\end{asy}
\item In trapezoid $ABCD, AD//BC, AD+BC=AB, F$ is the midpoint of $CD$. Prove: The angle bisectors of $\angle A$ and $\angle B$ intersect at $F$.
%problem 19, graph

\begin{asy}
 /* Geogebra to Asymptote conversion, documentation at artofproblemsolving.com/Wiki go to User:Azjps/geogebra */
import graph; size(4.916993954187907cm); 
real labelscalefactor = 0.5; /* changes label-to-point distance */
pen dps = linewidth(0.7) + fontsize(10); defaultpen(dps); /* default pen style */ 
pen dotstyle = black; /* point style */ 
real xmin = -3.29956, xmax = 1.6174339541879075, ymin = -2.0802067899929093, ymax = 1.7756096310348624;  /* image dimensions */

 /* draw figures */
draw((-1.,0.5)--(-1.5001774939768113,-0.8015257888315626), linewidth(2.)); 
draw((-1.5001774939768113,-0.8015257888315626)--(-0.55,-0.16), linewidth(2.)); 
draw((-0.55,-0.16)--(-1.,0.5), linewidth(2.)); 
draw((-0.62,0.5)--(-1.,0.5), linewidth(2.)); 
draw((-1.5001774939768113,-0.8015257888315626)--(-0.46,-0.8), linewidth(2.)); 
draw((-0.46,-0.8)--(-0.62,0.5), linewidth(2.)); 
 /* dots and labels */
dot((-1.,0.5),dotstyle); 
label("$A$", (-1.1178911007557677,0.5628400216859564), NE * labelscalefactor); 
dot((-1.5001774939768113,-0.8015257888315626),dotstyle); 
label("$B$", (-1.5660889410838879,-0.9663042683626641), NE * labelscalefactor); 
dot((-0.62,0.5),dotstyle); 
label("$D$", (-0.5905995238991556,0.5628400216859564), NE * labelscalefactor); 
dot((-0.55,-0.16),dotstyle); 
label("$F$", (-0.4917323532385408,-0.18854984497586574), NE * labelscalefactor); 
dot((-0.46,-0.8),dotstyle); 
label("$C$", (-0.4785500638171255,-0.9333485724564439), NE * labelscalefactor); 
clip((xmin,ymin)--(xmin,ymax)--(xmax,ymax)--(xmax,ymin)--cycle); 
 /* end of picture */
\end{asy}

\item In $\triangle ABC, AC=BC, \angle B=2\angle C$. Prove: $AC^2=AB^2+AC\cdot AB$.

 \end{enumerate}

 


\end{document}
