\documentclass[12pt]{article}

% Use this form to include EPS (latex) or PDF (pdflatex) files:
%\usepackage{asymptote}

% Use this form with latex or pdflatex to include inline LaTeX code by default:
\usepackage[inline]{asymptote}

\usepackage{fourier}
% Use this form with latex or pdflatex to create PDF attachments by default:
%\usepackage[attach]{asymptote}

% Enable this line to support the attach option:
%\usepackage[dvips]{attachfile2}

\begin{document}

% Optional subdirectory for latex files (no spaces):
\def\asylatexdir{}
% Optional subdirectory for asy files (no spaces):
\def\asydir{}

\begin{asydef}
// Global Asymptote definitions can be put here.
import three;
usepackage("bm");
texpreamble("\def\V#1{\bm{#1}}");
// One can globally override the default toolbar settings here:
// settings.toolbar=true;
\end{asydef}



\section*{Exercises}

\begin{enumerate}

\item Two circles intersect at $A$ and $B$, $AC$ and $AD$ are the diameters of two circles respectively, prove: $C, B$, and  $D$ are collinear.
 
\item Two circles  intersect at $A$ and $B$, $AD$ and $BF$ are the chords of the two circles, and   intersect with the other circles at $C$ and $E$ respectively, prove: $CF//DE$.

\item Two circles are tangent at $P$, one chord $AB$ of the first circle tangents the second circle at $C$. Assume the extension of $AP$ intersects the second circle at $D$, prove: $\angle BPC=\angle CPD$.

\item Given a half circle $O$, $AB$ is its diameter, $C$ is a point on this half circle, $CD\bot AB$ at $D$, $\odot P$ tangents $\odot O$ externally at $E$, tangents the line $CD$ at $F$, and $A, E$ are in the
same side of $CD$, prove: $A, E, F$ are collinear.

\end{enumerate} 

\section*{Exercises set 1}

 \begin{enumerate}
 \item Assume $AD$ is the median of $\triangle ABC$, $AE$ is the median of $\triangle ABD$, and $BA=BD$, prove: $AC=2AE$.
 %probem 1 graph
 
  \begin{asy}
    /* Geogebra to Asymptote conversion, documentation at artofproblemsolving.com/Wiki go to User:Azjps/geogebra */
import graph; size(4.cm); 
real labelscalefactor = 0.5; /* changes label-to-point distance */
pen dps = linewidth(0.7) + fontsize(10); defaultpen(dps); /* default pen style */ 
pen dotstyle = black; /* point style */ 
real xmin = -3.99399, xmax = 7.883700224892093, ymin = -8.713929985012639, ymax = 7.253134080690438;  /* image dimensions */

 /* draw figures */
draw((0.88,0.64)--(-1.8,-1.62), linewidth(2.)); 
draw((-1.8,-1.62)--(4.74,-1.58), linewidth(2.)); 
draw((4.74,-1.58)--(0.88,0.64), linewidth(2.)); 
draw((1.47,-1.6)--(0.88,0.64), linewidth(2.)); 
draw((-0.165,-1.61)--(0.88,0.64), linewidth(2.)); 
 /* dots and labels */
dot((0.88,0.64),dotstyle); 
label("$A$", (0.9584725332162531,0.8416488960065766), NE * labelscalefactor); 
dot((-1.8,-1.62),dotstyle); 
label("$B$", (-1.8773773820694024,-2.1585973763134354), NE * labelscalefactor); 
dot((4.74,-1.58),dotstyle); 
label("$C$", (4.513559745856966,-2.0147499522980925), NE * labelscalefactor); 
dot((1.47,-1.6),linewidth(4.pt) + dotstyle); 
label("$D$", (1.513312734033012,-2.055849216302476), NE * labelscalefactor); 
dot((-0.165,-1.61),dotstyle); 
label("$E$", (-0.27450569082098836,-2.1380477443112436), NE * labelscalefactor); 
clip((xmin,ymin)--(xmin,ymax)--(xmax,ymax)--(xmax,ymin)--cycle); 
 /* re-scale y/x */
currentpicture = yscale(1.115830390861128) * currentpicture; 
 /* end of picture */
 \end{asy} 
 
 
 \item Prove the perimeter of a triangle is greater than the sum of the three medians.
 
 \item Given a triangle $ABC$, assume $P$ is a point on the exterior angle bisector of $A$, prove: $PB+PC > AB+AC$.
% problem 3 , graph

 \begin{asy}
  /* Geogebra to Asymptote conversion, documentation at artofproblemsolving.com/Wiki go to User:Azjps/geogebra */
import graph; size(6.cm); 
real labelscalefactor = 0.5; /* changes label-to-point distance */
pen dps = linewidth(0.7) + fontsize(10); defaultpen(dps); /* default pen style */ 
pen dotstyle = black; /* point style */ 
real xmin = -12.65228500381642, xmax = 13.786436538120924, ymin = -10.810725412964283, ymax = 10.38936399682446;  /* image dimensions */

 /* draw figures */
draw((-0.6198018252876607,5.1780164070308325)--(-5.340027652692367,0.048522444302025035), linewidth(2.)); 
draw((-5.340027652692367,0.048522444302025035)--(2.436066918465675,0.0758069866569655), linewidth(2.)); 
draw((2.436066918465675,0.0758069866569655)--(-1.730923879399975,3.970554290423232), linewidth(2.)); 
draw((-1.730923879399975,3.970554290423232)--(2.46,4.13), linewidth(2.)); 
draw((-5.340027652692367,0.048522444302025035)--(2.46,4.13), linewidth(2.)); 
draw((2.46,4.13)--(2.436066918465675,0.0758069866569655), linewidth(2.)); 
 /* dots and labels */
dot((-0.6198018252876607,5.1780164070308325),dotstyle); 
label("$D$", (-0.5106636558678973,5.450861830580236), NE * labelscalefactor); 
dot((-5.340027652692367,0.048522444302025035),dotstyle); 
label("$B$", (-5.831149415081295,-0.49716840279678587), NE * labelscalefactor); 
dot((2.436066918465675,0.0758069866569655),dotstyle); 
label("$C$", (2.245075121981093,-0.5790220298616072), NE * labelscalefactor); 
dot((-1.730923879399975,3.970554290423232),dotstyle); 
label("$A$", (-2.2841589089390295,4.304911051672736), NE * labelscalefactor); 
dot((2.46,4.13),dotstyle); 
label("$P$", (2.572489630240379,4.414049221092498), NE * labelscalefactor); 
clip((xmin,ymin)--(xmin,ymax)--(xmax,ymax)--(xmax,ymin)--cycle); 
 /* re-scale y/x */
currentpicture = yscale(0.9976833976833988) * currentpicture; 
 /* end of picture */
 \end{asy}
 
 \item For a right triangle $ABC$, assume $AB$ is the hypotenuse, the perpendicular bisector $ME$ of $AB$ intersects the angle bisector of $C$ at $E$. Prove: $MC=ME$.
 %problem 4 graph
 
 \begin{asy}
  /* Geogebra to Asymptote conversion, documentation at artofproblemsolving.com/Wiki go to User:Azjps/geogebra */
import graph; size(6.cm); 
real labelscalefactor = 0.5; /* changes label-to-point distance */
pen dps = linewidth(0.7) + fontsize(10); defaultpen(dps); /* default pen style */ 
pen dotstyle = black; /* point style */ 
real xmin = -4.232814273524748, xmax = 7.719872903505422, ymin = -4.916564176694168, ymax = 3.8202809741350294;  /* image dimensions */
pen qqwuqq = rgb(0.,0.39215686274509803,0.); 

draw((2.958519252677728,2.218889439508296)--(3.139629813169432,1.9774086921860237)--(3.381110560491704,2.1585192526777277)--(3.2,2.4)--cycle, linewidth(2.) + qqwuqq); 
 /* draw figures */
draw((0.,0.)--(5.,0.), linewidth(2.)); 
draw((0.,0.)--(3.2,2.4), linewidth(2.)); 
draw((3.2,2.4)--(5.,0.), linewidth(2.)); 
draw((2.5,-2.5)--(3.2,2.4), linewidth(2.)); 
draw((2.5,-2.5)--(2.5,0.), linewidth(2.)); 
draw((2.329554410482433,0.)--(2.3258458953944197,-0.1928921158747807), linewidth(2.)); 
draw((2.5,-0.1928921158747805)--(2.3258458953944197,-0.1928921158747807), linewidth(2.)); 
 /* dots and labels */
dot((0.,0.),dotstyle); 
label("$A$", (-0.3197321619970134,-0.34893014832906627), NE * labelscalefactor); 
dot((5.,0.),dotstyle); 
label("$B$", (5.044747678170099,-0.3062419798396728), NE * labelscalefactor); 
dot((3.2,2.4),dotstyle); 
label("$C$", (3.20915643312618,2.596553477439083), NE * labelscalefactor); 
dot((2.5,-2.5),dotstyle); 
label("$E$", (2.4407694003170977,-2.8248439207138873), NE * labelscalefactor); 
dot((2.5,0.),linewidth(2.pt) + dotstyle); 
label("$M$", (2.0992640524019497,0.1348690945507264), NE * labelscalefactor); 
dot((2.329554410482433,0.),linewidth(1.pt) + dotstyle); 
dot((2.3258458953944197,-0.1928921158747807),linewidth(1.pt) + dotstyle); 
dot((2.5,-0.1928921158747805),linewidth(1.pt) + dotstyle); 
clip((xmin,ymin)--(xmin,ymax)--(xmax,ymax)--(xmax,ymin)--cycle); 
 /* re-scale y/x */
currentpicture = yscale(0.9804560260586318) * currentpicture; 
 /* end of picture */
 \end{asy}
 \item For an isosceles triangle $ABC$ with $AB=AC$, $CX$ is the altitude on $AB$, $XP\bot BC$ at $P$. Prove: $AB^2=PA^2+PX^2$.
 %problem 5, graph
 
 \begin{asy}
  /* Geogebra to Asymptote conversion, documentation at artofproblemsolving.com/Wiki go to User:Azjps/geogebra */
import graph; size(6.cm); 
real labelscalefactor = 0.5; /* changes label-to-point distance */
pen dps = linewidth(0.7) + fontsize(10); defaultpen(dps); /* default pen style */ 
pen dotstyle = black; /* point style */ 
real xmin = -3.6728739438707847, xmax = 3.8390868078546636, ymin = -1.0029185521767334, ymax = 3.5826960432540624;  /* image dimensions */
pen qqwuqq = rgb(0.,0.39215686274509803,0.); 

draw((-0.6793649258151686,1.7273114497069317)--(-0.609310950991452,1.834432643766279)--(-0.7164321450507992,1.9044866185899956)--(-0.7864861198745158,1.7973654245306483)--cycle, linewidth(2.) + qqwuqq); 
draw((-0.7864861198745158,0.12799417801330745)--(-0.9144802978878233,0.12799417801330748)--(-0.9144802978878233,0.)--(-0.7864861198745158,0.)--cycle, linewidth(2.) + qqwuqq); 
 /* draw figures */
draw((-1.9619079708420344,0.)--(0.,3.), linewidth(2.)); 
draw((0.,3.)--(1.9619079708420344,0.), linewidth(2.)); 
draw((0.,3.)--(-0.7864861198745158,0.), linewidth(2.)); 
draw((-1.9619079708420344,0.)--(1.9619079708420344,0.), linewidth(2.)); 
draw((-0.7864861198745158,1.7973654245306483)--(-0.7864861198745158,0.), linewidth(2.)); 
draw((-0.7864861198745158,1.7973654245306483)--(1.9619079708420344,0.), linewidth(2.)); 
 /* dots and labels */
dot((-1.9619079708420344,0.),linewidth(4.pt) + dotstyle); 
label("$B$", (-2.1885829037708167,-0.22457081163650625), NE * labelscalefactor); 
dot((1.9619079708420344,0.),linewidth(4.pt) + dotstyle); 
label("$C$", (1.9927735628522643,-0.1823348877312226), NE * labelscalefactor); 
dot((0.,3.),linewidth(4.pt) + dotstyle); 
label("$A$", (0.025786249549054466,3.045696439315456), NE * labelscalefactor); 
dot((-0.7864861198745158,1.7973654245306483),linewidth(4.pt) + dotstyle); 
label("$X$", (-1.2050892471192118,1.7182816880065415), NE * labelscalefactor); 
dot((-0.7864861198745158,0.),linewidth(4.pt) + dotstyle); 
label("$P$", (-0.9034040763671857,-0.2426719218816278), NE * labelscalefactor); 
clip((xmin,ymin)--(xmin,ymax)--(xmax,ymax)--(xmax,ymin)--cycle); 
 /* re-scale y/x */
currentpicture = yscale(1.0566118421052633) * currentpicture; 
 /* end of picture */
 \end{asy}
 
 \item Prove the length of the diagonal of a rectangle is longer than any line segment between the two opposite sides.
 
 \item For the square $ABCD$, assume $E$ is the midpoint of $CD$, $BF\bot AE$ at $F$, prove: $CF=CB$.
 
 %problem 7, graph is here
 \begin{asy}
  /* Geogebra to Asymptote conversion, documentation at artofproblemsolving.com/Wiki go to User:Azjps/geogebra */
import graph; size(6.cm); 
real labelscalefactor = 0.5; /* changes label-to-point distance */
pen dps = linewidth(0.7) + fontsize(10); defaultpen(dps); /* default pen style */ 
pen dotstyle = black; /* point style */ 
real xmin = -3.018132430655583, xmax = 10.766509713584558, ymin = -5.300893330630993, ymax = 6.578554370909291;  /* image dimensions */
pen zzttqq = rgb(0.6,0.2,0.); pen qqwuqq = rgb(0.,0.39215686274509803,0.); 

draw((0.,0.)--(4.12634672946888,0.)--(4.12634672946888,4.126346729468878)--(0.,4.12634672946888)--cycle, linewidth(2.) + zzttqq); 
draw((0.6923703644725991,1.3847407289451976)--(0.9581683273149537,1.2518417475240204)--(1.091067308736131,1.517639710366375)--(0.8252693458937764,1.6505386917875522)--cycle, linewidth(2.) + qqwuqq); 
 /* draw figures */
draw((0.,0.)--(4.12634672946888,0.), linewidth(2.) + zzttqq); 
draw((4.12634672946888,0.)--(4.12634672946888,4.126346729468878), linewidth(2.) + zzttqq); 
draw((4.12634672946888,4.126346729468878)--(0.,4.12634672946888), linewidth(2.) + zzttqq); 
draw((0.,4.12634672946888)--(0.,0.), linewidth(2.) + zzttqq); 
draw((0.,0.)--(2.0631733647344404,4.126346729468879), linewidth(2.)); 
draw((4.12634672946888,4.126346729468878)--(0.8252693458937764,1.6505386917875522), linewidth(2.)); 
draw((4.12634672946888,0.)--(0.8252693458937764,1.6505386917875522), linewidth(2.)); 
 /* dots and labels */
dot((0.,0.),linewidth(4.pt) + dotstyle); 
label("$A$", (-0.4965515506116546,-0.4818720932137077), NE * labelscalefactor); 
dot((4.12634672946888,0.),dotstyle); 
label("$B$", (4.12634672946888,-0.5799335718820827), NE * labelscalefactor); 
dot((4.12634672946888,4.126346729468878),linewidth(4.pt) + dotstyle); 
label("$C$", (4.182381860136522,4.239087665535202), NE * labelscalefactor); 
dot((0.,4.12634672946888),linewidth(4.pt) + dotstyle); 
label("$D$", (0.04979097339786314,4.239087665535202), NE * labelscalefactor); 
dot((2.0631733647344404,4.126346729468879),linewidth(4.pt) + dotstyle); 
label("$E$", (2.1230908081006485,4.239087665535202), NE * labelscalefactor); 
dot((0.8252693458937764,1.6505386917875522),linewidth(4.pt) + dotstyle); 
label("$F$", (0.30194906140225597,1.5634101761552561), NE * labelscalefactor); 
clip((xmin,ymin)--(xmin,ymax)--(xmax,ymax)--(xmax,ymin)--cycle); 
 /* re-scale y/x */
currentpicture = yscale(0.9998584905660378) * currentpicture; 
 /* end of picture */
   \end{asy}
 \item For an isosceles triangle $ABC, AB=AC$, the circle which takes $AB$ as a diameter intersects $AC$ and $BC$ at $E$ and $D$ respectively. Make $DF\bot AC$ at $F$, prove: $DF^2=EF\cdot FA$.
 
 %graphi is here, problem 8
 \vskip 0.1 cm
  
 \begin{asy}
  /* Geogebra to Asymptote conversion, documentation at artofproblemsolving.com/Wiki go to User:Azjps/geogebra */
import graph; size(6.cm); 
real labelscalefactor = 0.5; /* changes label-to-point distance */
pen dps = linewidth(0.7) + fontsize(10); defaultpen(dps); /* default pen style */ 
pen dotstyle = black; /* point style */ 
real xmin = -4.298031539044008, xmax = 5.080637348048336, ymin = -3.9943088418843256, ymax = 2.6387694220395344;  /* image dimensions */
pen qqwuqq = rgb(0.,0.39215686274509803,0.); 

draw((2.345750090152916,-0.9550816307571283)--(2.434806317125606,-1.1093315406042126)--(2.5890562269726902,-1.020275313631523)--(2.5,-0.8660254037844387)--cycle, linewidth(2.) + qqwuqq); 
 /* draw figures */
draw(shift((0.,0.)) * scale(2., 2.)*unitcircle, linewidth(2.)); 
draw((1.,1.7320508075688772)--(3.,-1.7320508075688772), linewidth(2.)); 
draw((1.,-1.7320508075688774)--(1.,1.7320508075688772), linewidth(2.)); 
draw((1.,-1.7320508075688774)--(2.5,-0.8660254037844387), linewidth(2.)); 
draw((-1.,-1.7320508075688776)--(3.,-1.7320508075688772), linewidth(2.)); 
draw((1.,1.7320508075688772)--(-1.,-1.7320508075688776), linewidth(2.)); 
draw((1.,1.7320508075688772)--(1.,1.7320508075688772), linewidth(2.)); 
 /* dots and labels */
dot((1.,1.7320508075688772),linewidth(4.pt) + dotstyle); 
label("$A$", (1.0756014848943096,1.899894881248016), NE * labelscalefactor); 
dot((1.,-1.7320508075688774),linewidth(4.pt) + dotstyle); 
label("$D$", (0.7901272304975866,-2.096744680306107), NE * labelscalefactor); 
dot((-1.,-1.7320508075688776),linewidth(4.pt) + dotstyle); 
label("$B$", (-1.35932597907774,-2.004385362707167), NE * labelscalefactor); 
dot((3.,-1.7320508075688772),linewidth(4.pt) + dotstyle); 
label("$C$", (2.704483995275612,-2.0631594739064925), NE * labelscalefactor); 
dot((2.,0.),linewidth(4.pt) + dotstyle); 
label("$E$", (2.032779867283322,0.06950113246902691), NE * labelscalefactor); 
dot((2.5,-0.8660254037844387),linewidth(4.pt) + dotstyle); 
label("$F$", (2.5365579632775397,-0.795317932321046), NE * labelscalefactor); 
clip((xmin,ymin)--(xmin,ymax)--(xmax,ymax)--(xmax,ymin)--cycle); 
 /* re-scale y/x */
currentpicture = yscale(0.949685654008439) * currentpicture; 
 /* end of picture */
 \end{asy}
 \item Prove: For a triangle, the symmetric points of the orthocenter along the sides are on the circumcircle.
 
 \item In the following figure, $AB$ is a diameter of $\odot O$, $AT$ is a tangent line of $\odot O$, $P$ is on the extension of $BM$ such that $PT\bot AT, PT=PM$. Prove: $PB=AB$.
 
 %graph here problem 10
 
 \begin{asy}
   /* Geogebra to Asymptote conversion, documentation at artofproblemsolving.com/Wiki go to User:Azjps/geogebra */
import graph; size(6.cm); 
real labelscalefactor = 0.5; /* changes label-to-point distance */
pen dps = linewidth(0.7) + fontsize(10); defaultpen(dps); /* default pen style */ 
pen dotstyle = black; /* point style */ 
real xmin = -4.473104296154636, xmax = 6.243997739846385, ymin = -4.581187822491017, ymax = 4.297987097254904;  /* image dimensions */
pen qqwuqq = rgb(0.,0.39215686274509803,0.); 

draw((-2.,2.336565393128929)--(-1.7365653931289289,2.336565393128929)--(-1.7365653931289289,2.6)--(-2.,2.6)--cycle, linewidth(2.) + qqwuqq); 
 /* draw figures */
draw(shift((0.,0.)) * scale(2., 2.)*unitcircle, linewidth(2.)); 
draw((-2.,0.)--(2.,0.), linewidth(2.)); 
draw((-2.,0.)--(-2.,2.6), linewidth(2.)); 
draw((-2.,2.6)--(-1.03,2.6), linewidth(2.)); 
draw((-1.03,2.6)--(2.,0.), linewidth(2.)); 
 /* dots and labels */
dot((-2.,0.),linewidth(4.pt) + dotstyle); 
label("$A$", (-1.9521637245344654,0.10055895337501349), NE * labelscalefactor); 
dot((2.,0.),linewidth(4.pt) + dotstyle); 
label("$B$", (2.046569595966495,0.10055895337501349), NE * labelscalefactor); 
dot((-2.,2.6),dotstyle); 
label("$T$", (-1.9521637245344654,2.7208469366846493), NE * labelscalefactor); 
dot((-1.03,2.6),dotstyle); 
label("$P$", (-0.9835264605621832,2.7208469366846493), NE * labelscalefactor); 
dot((0.,0.),dotstyle); 
label("$O$", (0.047202935716014735,0.1253958062973797), NE * labelscalefactor); 
dot((-0.3037344190102191,1.976801811691937),linewidth(4.pt) + dotstyle); 
label("$M$", (-0.2508392993523798,2.0750887607031276), NE * labelscalefactor); 
clip((xmin,ymin)--(xmin,ymax)--(xmax,ymax)--(xmax,ymin)--cycle); 
 /* re-scale y/x */
currentpicture = yscale(0.9655944055944056) * currentpicture; 
 /* end of picture */
 \end{asy}
 \item In the following figure, $AB$ is a diameter of $\odot O$, $P$ is a point on the circle, $Q$ is the midpoint of the arc $\widearc{BP}$, the tangent line $QH$ intersects $AP$ at $H$, prove: $QH\bot AP$.
 %problem 11 graph
 
 \begin{asy}
  /* Geogebra to Asymptote conversion, documentation at artofproblemsolving.com/Wiki go to User:Azjps/geogebra */
import graph; size(6.cm); 
real labelscalefactor = 0.5; /* changes label-to-point distance */
pen dps = linewidth(0.7) + fontsize(10); defaultpen(dps); /* default pen style */ 
pen dotstyle = black; /* point style */ 
real xmin = -4.473104296154635, xmax = 5.958373931239169, ymin = -2.9667923825372107, ymax = 4.297987097254902;  /* image dimensions */

 /* draw figures */
draw(shift((0.,0.)) * scale(2., 2.)*unitcircle, linewidth(2.)); 
draw((-2.,0.)--(2.,0.), linewidth(2.)); 
draw((1.6416256586779208,1.1423945013743206)--(0.9890930602927794,2.080086564268647), linewidth(2.)); 
draw((-2.,0.)--(0.9890930602927794,2.080086564268647), linewidth(2.)); 
 /* dots and labels */
dot((-2.,0.),linewidth(4.pt) + dotstyle); 
label("$A$", (-2.349553371292325,-0.09813587000391474), NE * labelscalefactor); 
dot((2.,0.),linewidth(4.pt) + dotstyle); 
label("$B$", (2.15833543411714,-0.18506485523219643), NE * labelscalefactor); 
dot((0.,0.),dotstyle); 
label("$O$", (0.0472029357160133,0.12539580629738104), NE * labelscalefactor); 
dot((0.6949348032297175,1.8753841257886539),linewidth(4.pt) + dotstyle); 
label("$P$", (0.5687768470857035,2.0005782019360288), NE * labelscalefactor); 
dot((1.6416256586779208,1.1423945013743206),linewidth(4.pt) + dotstyle); 
label("$Q$", (1.6864352285921824,1.24305418780386), NE * labelscalefactor); 
dot((0.9890930602927794,2.080086564268647),linewidth(4.pt) + dotstyle); 
label("$H$", (1.0406770526106612,2.1744361723925922), NE * labelscalefactor); 
clip((xmin,ymin)--(xmin,ymax)--(xmax,ymax)--(xmax,ymin)--cycle); 
 /* re-scale y/x */
currentpicture = yscale(1.005128205128205) * currentpicture; 
 /* end of picture */
   \end{asy}
 
\item In the following figure, two circles  tangent internally at $P$, a secant intersects the two circles at $A, B, C, D$, prove: $\angle APB=\angle CPD$.
%problem 12, graph

\begin{asy}
 /* Geogebra to Asymptote conversion, documentation at artofproblemsolving.com/Wiki go to User:Azjps/geogebra */
import graph; size(6.cm); 
real labelscalefactor = 0.5; /* changes label-to-point distance */
pen dps = linewidth(0.7) + fontsize(10); defaultpen(dps); /* default pen style */ 
pen dotstyle = black; /* point style */ 
real xmin = -4.4731042961546335, xmax = 5.958373931239167, ymin = -2.966792382537209, ymax = 4.297987097254902;  /* image dimensions */

 /* draw figures */
draw(shift((0.,0.)) * scale(2., 2.)*unitcircle, linewidth(2.)); 
draw(shift((-0.7599947842608878,-0.358922825688759)) * scale(1.144113882562022, 1.144113882562022)*unitcircle, linewidth(2.)); 
draw((-1.9670045595430818,0.361791463051198)--(1.974168812869036,0.3204020884689435), linewidth(2.)); 
draw((-1.795492987832248,-0.8454822225475906)--(-1.9670045595430818,0.361791463051198), linewidth(2.)); 
draw((-1.795492987832248,-0.8454822225475906)--(-1.6512502196468668,0.3584754773243538), linewidth(2.)); 
draw((-1.795492987832248,-0.8454822225475906)--(0.14613035480935482,0.3395997642223174), linewidth(2.)); 
draw((-1.795492987832248,-0.8454822225475906)--(1.974168812869036,0.3204020884689435), linewidth(2.)); 
 /* dots and labels */
dot((-0.7599947842608878,-0.358922825688759),dotstyle); 
dot((-1.9670045595430818,0.361791463051198),dotstyle); 
label("$A$", (-2.250205959602859,0.26199849737039577), NE * labelscalefactor); 
dot((1.974168812869036,0.3204020884689435),dotstyle); 
label("$D$", (2.0217327430441254,0.44827489428814216), NE * labelscalefactor); 
dot((-1.795492987832248,-0.8454822225475906),dotstyle); 
label("$P$", (-2.0515111362239296,-1.1785389721268433), NE * labelscalefactor); 
dot((-1.6512502196468668,0.3584754773243538),linewidth(4.pt) + dotstyle); 
label("$B$", (-1.5423556513154228,0.1378142327585648), NE * labelscalefactor); 
dot((0.14613035480935482,0.3395997642223174),linewidth(4.pt) + dotstyle); 
label("$C$", (0.19622405325021042,0.4358564678269591), NE * labelscalefactor); 
clip((xmin,ymin)--(xmin,ymax)--(xmax,ymax)--(xmax,ymin)--cycle); 
 /* re-scale y/x */
currentpicture = yscale(1.0290598290598292) * currentpicture; 
 /* end of picture */
\end{asy}
\item In the following figure, two circles  tangent externally at $P$, a secant intersects the two circles at $A, B, C, D$, prove: $\angle APD+\angle BPC=180^\circ$.
%problem 13 , graph

\begin{asy}
 /* Geogebra to Asymptote conversion, documentation at artofproblemsolving.com/Wiki go to User:Azjps/geogebra */
import graph; size(6.cm); 
real labelscalefactor = 0.5; /* changes label-to-point distance */
pen dps = linewidth(0.7) + fontsize(10); defaultpen(dps); /* default pen style */ 
pen dotstyle = black; /* point style */ 
real xmin = -4.473104296154634, xmax = 8.678009326238264, ymin = -4.432166704956815, ymax = 4.297987097254902;  /* image dimensions */

 /* draw figures */
draw(shift((0.,0.)) * scale(1.4142135623730951, 1.4142135623730951)*unitcircle, linewidth(2.)); 
draw(shift((-2.3743902242146904,0.)) * scale(0.9601766618415952, 0.9601766618415952)*unitcircle, linewidth(2.)); 
draw((-1.4142135623730951,0.)--(-3.2302131999452177,0.43532316289971834), linewidth(2.)); 
draw((-1.4142135623730951,0.)--(-1.710222996915708,0.693412659335875), linewidth(2.)); 
draw((-1.4142135623730951,0.)--(-1.17704306258552,0.7839449144036201), linewidth(2.)); 
draw((-1.4142135623730951,0.)--(0.8523121072560532,1.1285229602559028), linewidth(2.)); 
draw((0.8523121072560532,1.1285229602559028)--(-3.2302131999452177,0.43532316289971834), linewidth(2.)); 
 /* dots and labels */
dot((-1.4142135623730951,0.),linewidth(4.pt) + dotstyle); 
label("$P$", (-1.3684976808588603,0.10055895337501483), NE * labelscalefactor); 
dot((0.8523121072560532,1.1285229602559028),linewidth(4.pt) + dotstyle); 
label("$D$", (0.9040743615376461,1.2306357613426768), NE * labelscalefactor); 
dot((-1.17704306258552,0.7839449144036201),linewidth(4.pt) + dotstyle); 
label("$C$", (-1.194639710402297,0.4855301736716909), NE * labelscalefactor); 
dot((-3.2302131999452177,0.43532316289971834),linewidth(4.pt) + dotstyle); 
label("$A$", (-3.715580282022465,0.1502326592197472), NE * labelscalefactor); 
dot((-1.710222996915708,0.693412659335875),linewidth(4.pt) + dotstyle); 
label("$B$", (-1.716213621771987,0.7835724087400853), NE * labelscalefactor); 
clip((xmin,ymin)--(xmin,ymax)--(xmax,ymax)--(xmax,ymin)--cycle); 
 /* re-scale y/x */
currentpicture = yscale(1.0042674253200568) * currentpicture; 
 /* end of picture */
\end{asy}
\item Two circles intersect at $A,B$, the line passes $A$ intersect the two circles at $C$ and $D$, the tangent lines at $C$ and $D$ intersect at $P$. Prove: $B, C, P, D$ are cyclic.
%problem 14, graph

\begin{asy}
 /* Geogebra to Asymptote conversion, documentation at artofproblemsolving.com/Wiki go to User:Azjps/geogebra */
import graph; size(6.cm); 
real labelscalefactor = 0.5; /* changes label-to-point distance */
pen dps = linewidth(0.7) + fontsize(10); defaultpen(dps); /* default pen style */ 
pen dotstyle = black; /* point style */ 
real xmin = -4.435849016771084, xmax = 2.928277874710491, ymin = -2.9667923825372085, ymax = 4.297987097254902;  /* image dimensions */

 /* draw figures */
draw(shift((0.,0.)) * scale(1.4142135623730951, 1.4142135623730951)*unitcircle, linewidth(2.)); 
draw(shift((-2.,0.)) * scale(1., 1.)*unitcircle, linewidth(2.)); 
draw((-2.8551526319614684,0.5183762880855693)--(1.118100142291255,0.8659399931913732), linewidth(2.)); 
draw((1.118100142291255,0.8659399931913732)--(-0.9925018830610111,3.591145484820686), linewidth(2.)); 
draw((-0.9925018830610111,3.591145484820686)--(-2.8551526319614684,0.5183762880855693), linewidth(2.)); 
 /* dots and labels */
dot((-2.8551526319614684,0.5183762880855693),dotstyle); 
label("$C$", (-3.119495811885676,0.6221328647447055), NE * labelscalefactor); 
dot((1.118100142291255,0.8659399931913732),dotstyle); 
label("$D$", (1.1648613172224918,0.9946856585801983), NE * labelscalefactor); 
dot((-1.25,0.6614378277661476),linewidth(4.pt) + dotstyle); 
label("$A$", (-1.3809161073200427,0.8705013939683673), NE * labelscalefactor); 
dot((-1.25,-0.6614378277661477),linewidth(4.pt) + dotstyle); 
label("$B$", (-1.1201291516351977,-0.7687308989078011), NE * labelscalefactor); 
dot((-0.9925018830610111,3.591145484820686),linewidth(4.pt) + dotstyle); 
label("$P$", (-0.9462711811786343,3.68948420065693), NE * labelscalefactor); 
clip((xmin,ymin)--(xmin,ymax)--(xmax,ymax)--(xmax,ymin)--cycle); 
 /* re-scale y/x */
currentpicture = yscale(0.9967806267806267) * currentpicture; 
 /* end of picture */
\end{asy}

\item In $\triangle ABC, \angle C=90^\circ, CD $ is the altitude, the circle which takes $CD$ as a diameter intersects $AC$ and $BC$ at $E$ and $F$ respectively, prove: $\frac{BF}{AE}=\frac{BC^3}{AC^3}$.
%problem 15, graph

\begin{asy}
 /* Geogebra to Asymptote conversion, documentation at artofproblemsolving.com/Wiki go to User:Azjps/geogebra */
import graph; size(6.cm); 
real labelscalefactor = 0.5; /* changes label-to-point distance */
pen dps = linewidth(0.7) + fontsize(10); defaultpen(dps); /* default pen style */ 
pen dotstyle = black; /* point style */ 
real xmin = -8.82112926636752, xmax = 7.88774112496929, ymin = -7.83284533970906, ymax = 5.719300352287245;  /* image dimensions */
pen qqwuqq = rgb(0.,0.39215686274509803,0.); 

draw((0.3848519726280579,-2.2632218759301237)--(0.384851972628058,-1.8783699033020658)--(0.,-1.8783699033020658)--(0.,-2.2632218759301237)--cycle, linewidth(2.) + qqwuqq); 
 /* draw figures */
draw(shift((0.,0.)) * scale(2.2632218759301237, 2.2632218759301237)*unitcircle, linewidth(2.)); 
draw((3.827933968690265,-2.2632218759301237)--(-5.35241548217325,-2.2632218759301237), linewidth(2.)); 
draw((0.,2.2632218759301237)--(-5.35241548217325,-2.2632218759301237), linewidth(2.)); 
draw((0.,2.2632218759301237)--(3.827933968690265,-2.2632218759301237), linewidth(2.)); 
draw((0.,2.2632218759301237)--(0.,-2.2632218759301237), linewidth(2.)); 
 /* dots and labels */
dot((0.,-2.2632218759301237),linewidth(4.pt) + dotstyle); 
label("$D$", (-0.22177577506062704,-2.662347987340992), NE * labelscalefactor); 
dot((0.,2.2632218759301237),linewidth(4.pt) + dotstyle); 
label("$C$", (0.1410661444037988,2.6532861328128465), NE * labelscalefactor); 
dot((-5.35241548217325,-2.2632218759301237),linewidth(4.pt) + dotstyle); 
label("$A$", (-5.6099782791073505,-2.6986321792874346), NE * labelscalefactor); 
dot((-2.2317988164195075,0.3758288210304701),linewidth(4.pt) + dotstyle); 
label("$E$", (-2.0904116603024203,0.004540120722537921), NE * labelscalefactor); 
dot((3.827933968690265,-2.2632218759301237),linewidth(4.pt) + dotstyle); 
label("$B$", (3.823911626967721,-2.662347987340992), NE * labelscalefactor); 
dot((2.2317988164195075,-0.37582882103046994),linewidth(4.pt) + dotstyle); 
label("$F$", (2.445112333002903,-0.3945859906883305), NE * labelscalefactor); 
clip((xmin,ymin)--(xmin,ymax)--(xmax,ymax)--(xmax,ymin)--cycle); 
 /* re-scale y/x */
currentpicture = yscale(0.9863453815261043) * currentpicture; 
 /* end of picture */
 \end{asy}
\item In the following $\triangle ABC, \angle B=3\angle C, AD $ is the angle bisector of $\angle A$, $BD\bot AD$. Prove: $BD=\frac12(AC-AB)$.

%problem 16, graph

\begin{asy}
 /* Geogebra to Asymptote conversion, documentation at artofproblemsolving.com/Wiki go to User:Azjps/geogebra */
import graph; size(6.cm); 
real labelscalefactor = 0.5; /* changes label-to-point distance */
pen dps = linewidth(0.7) + fontsize(10); defaultpen(dps); /* default pen style */ 
pen dotstyle = black; /* point style */ 
real xmin = -5.326227346015427, xmax = 6.0827691090678515, ymin = -3.6186913877569715, ymax = 5.664010874028451;  /* image dimensions */
pen qqwuqq = rgb(0.,0.39215686274509803,0.); 

draw((-1.0551700369516475,1.1194357623344144)--(-1.3559501992554777,0.9683182924052958)--(-1.2048327293263588,0.6675381301014655)--(-0.9040525670225286,0.8186556000305842)--cycle, linewidth(2.) + qqwuqq); 
 /* draw figures */
draw((-2.,3.)--(-2.5334827339227197,0.), linewidth(2.)); 
draw((-2.5334827339227197,0.)--(3.987396439929115,0.006301780035442572), linewidth(2.)); 
draw((-2.,3.)--(-0.9040525670225286,0.8186556000305842), linewidth(2.)); 
draw((-2.,3.)--(3.987396439929115,0.006301780035442572), linewidth(2.)); 
draw((-2.5334827339227197,0.)--(-0.9040525670225286,0.8186556000305842), linewidth(2.)); 
 /* dots and labels */
dot((-2.,3.),dotstyle); 
label("$A$", (-2.3589381593940364,3.3155670745130754), NE * labelscalefactor); 
dot((-2.5334827339227197,0.),dotstyle); 
label("$B$", (-2.7556349295285236,-0.5085897895833804), NE * labelscalefactor); 
dot((3.987396439929115,0.006301780035442572),linewidth(4.pt) + dotstyle); 
label("$C$", (4.210360354033072,-0.2864395983080676), NE * labelscalefactor); 
dot((-0.9040525670225286,0.8186556000305842),linewidth(4.pt) + dotstyle); 
label("$D$", (-0.724547466439949,0.4434824587393887), NE * labelscalefactor); 
clip((xmin,ymin)--(xmin,ymax)--(xmax,ymax)--(xmax,ymin)--cycle); 
 /* re-scale y/x */
currentpicture = yscale(0.9996353276353276) * currentpicture; 
 /* end of picture */
\end{asy}
\item In right triangle $ABC, \angle A=90^\circ, AD $ is the altitude on $BC, BF$ is the angle bisector of $\angle B$, $AD$ and $BF$ intersect at $E$, $EG//BC$. Prove: $CG=AF$.
%problem 17

\begin{asy} 
 /* Geogebra to Asymptote conversion, documentation at artofproblemsolving.com/Wiki go to User:Azjps/geogebra */
import graph; size(6.cm); 
real labelscalefactor = 0.5; /* changes label-to-point distance */
pen dps = linewidth(0.7) + fontsize(10); defaultpen(dps); /* default pen style */ 
pen dotstyle = black; /* point style */ 
real xmin = -2.607670792987398, xmax = 2.1096813156147505, ymin = -1.6646320172611788, ymax = 1.8011776951812162;  /* image dimensions */
pen qqwuqq = rgb(0.,0.39215686274509803,0.); 

draw((-0.10430183465775293,1.1695997854198394)--(-0.04469578920468061,1.0652979507620866)--(0.059606045453072305,1.1249039962151588)--(0.,1.2292058308729117)--cycle, linewidth(2.) + qqwuqq); 
draw((0.0020740713790168026,0.11882419399557959)--(-0.11805795189161868,0.11859980052319677)--(-0.11783355841923585,-0.001532222747438707)--(0.0022984648513996323,-0.0013078292750558847)--cycle, linewidth(2.) + qqwuqq); 
 /* draw figures */
draw((0.,1.2292058308729117)--(-2.1602868595185263,-0.005347301990555229), linewidth(2.)); 
draw((0.,1.2292058308729117)--(0.7024622228996065,0.), linewidth(2.)); 
draw((-2.1602868595185263,-0.005347301990555229)--(0.7024622228996065,0.), linewidth(2.)); 
draw((-2.1602868595185263,-0.005347301990555229)--(0.32664324876846257,0.657628067809562), linewidth(2.)); 
draw((0.,1.2292058308729117)--(0.0022984648513996323,-0.0013078292750558847), linewidth(2.)); 
draw((0.0012296842084459738,0.5708780712643213)--(0.3758189741311438,0.57157776306335), linewidth(2.)); 
 /* dots and labels */
dot((0.,1.2292058308729117),dotstyle); 
label("$A$", (-0.053618210779032144,1.2914997962926287), NE * labelscalefactor); 
dot((-2.1602868595185263,-0.005347301990555229),dotstyle); 
label("$B$", (-2.2565593515308158,-0.2262077248422765), NE * labelscalefactor); 
dot((0.7024622228996065,0.),linewidth(4.pt) + dotstyle); 
label("$C$", (0.7448771641464216,-0.1355983205954165), NE * labelscalefactor); 
dot((-0.07777904243781153,-0.001457405295097643),linewidth(1.pt) + dotstyle); 
dot((0.32664324876846257,0.657628067809562),linewidth(4.pt) + dotstyle); 
label("$F$", (0.3484610205664091,0.7025386686880386), NE * labelscalefactor); 
dot((0.0022984648513996323,-0.0013078292750558847),linewidth(4.pt) + dotstyle); 
label("$D$", (0.025665017936970354,0.04562048789830352), NE * labelscalefactor); 
dot((0.0012296842084459738,0.5708780712643213),linewidth(4.pt) + dotstyle); 
label("$E$", (-0.16687996608760713,0.5836138256140349), NE * labelscalefactor); 
dot((0.3758189741311438,0.57157776306335),linewidth(4.pt) + dotstyle); 
label("$G$", (0.43340733704784035,0.5213198601943185), NE * labelscalefactor); 
dot((-1.234553132863467,3.389492690391438),dotstyle); 
label("$B'$", (-2.607670792987398,1.8011776951812162), NE * labelscalefactor); 
dot((-0.001741007864099712,2.16127749509487),dotstyle); 
label("$B'_{1}$", (-2.607670792987398,1.8011776951812162), NE * labelscalefactor); 
clip((xmin,ymin)--(xmin,ymax)--(xmax,ymax)--(xmax,ymin)--cycle); 
 /* re-scale y/x */
currentpicture = yscale(0.9981481481481483) * currentpicture; 
 /* end of picture */
\end{asy}
\item In the following figure, $D$ and $E$ are the midpoints of $AB$ and $AC$ respectively,  $AB>AC$. $F$ is a point between $B$ and $D$ such that $BF=AE$.  $AH$ is the angle bisector of 
$\angle BAC$, $FH\bot AH$, and $FH$ intersects $BC$ at $M$. Prove: $BM=MC$.
%problem 18, graph

\begin{asy}
 /* Geogebra to Asymptote conversion, documentation at artofproblemsolving.com/Wiki go to User:Azjps/geogebra */
import graph; size(6.cm); 
real labelscalefactor = 0.5; /* changes label-to-point distance */
pen dps = linewidth(0.7) + fontsize(10); defaultpen(dps); /* default pen style */ 
pen dotstyle = black; /* point style */ 
real xmin = -4.665799584087608, xmax = 6.132149618605442, ymin = -3.9849785448935737, ymax = 1.7131476834404902;  /* image dimensions */
pen qqwuqq = rgb(0.,0.39215686274509803,0.); 

draw((-1.3259113520557209,-0.5537883842355402)--(-1.518375941227879,-0.4942637249981439)--(-1.5779006004652754,-0.686728314170302)--(-1.3854360112931172,-0.7462529734076983)--cycle, linewidth(2.) + qqwuqq); 
 /* draw figures */
draw((-1.,0.5)--(-3.556526183665778,-0.46167474859710544), linewidth(2.)); 
draw((-3.556526183665778,-0.46167474859710544)--(-0.30058754755411127,-0.4983333864943361), linewidth(2.)); 
draw((-0.30058754755411127,-0.4983333864943361)--(-1.,0.5), linewidth(2.)); 
draw((-1.,0.5)--(-1.3854360112931172,-0.7462529734076983), linewidth(2.)); 
draw((-1.3854360112931172,-0.7462529734076983)--(-2.99,-0.25), linewidth(2.)); 
 /* dots and labels */
dot((-1.,0.5),dotstyle); 
label("$A$", (-0.9620175356704668,0.5925161918681243), NE * labelscalefactor); 
dot((-3.556526183665778,-0.46167474859710544),dotstyle); 
label("$B$", (-3.944036928498627,-0.7845309799792745), NE * labelscalefactor); 
dot((-0.30058754755411127,-0.4983333864943361),dotstyle); 
label("$C$", (-0.24025488008148538,-0.7180528406487103), NE * labelscalefactor); 
dot((-2.2782630918328888,0.019162625701447278),linewidth(4.pt) + dotstyle); 
label("$D$", (-2.3200709534234187,0.10817546245972887), NE * labelscalefactor); 
dot((-0.6502937737770557,8.333067528319416e-4),linewidth(4.pt) + dotstyle); 
label("$E$", (-0.6106330849231996,0.07968483131805855), NE * labelscalefactor); 
dot((-2.99,-0.25),linewidth(4.pt) + dotstyle); 
label("$F$", (-3.0038461008235062,-0.1672339719097509), NE * labelscalefactor); 
dot((-1.3854360112931172,-0.7462529734076983),linewidth(4.pt) + dotstyle); 
label("$H$", (-1.4653520191733092,-1.0504435373015308), NE * labelscalefactor); 
dot((-2.2583190919285236,-0.4762912721285338),linewidth(4.pt) + dotstyle); 
label("$M$", (-2.215605305903961,-0.40465589809033686), NE * labelscalefactor); 
clip((xmin,ymin)--(xmin,ymax)--(xmax,ymax)--(xmax,ymin)--cycle); 
 /* re-scale y/x */
currentpicture = yscale(0.9790833333333333) * currentpicture; 
 /* end of picture */
\end{asy}
\item In trapezoid $ABCD, AD//BC, AD+BC=AB, F$ is the midpoint of $CD$. Prove: The angle bisectors of $\angle A$ and $\angle B$ intersect at $F$.
%problem 19, graph

\begin{asy}
 /* Geogebra to Asymptote conversion, documentation at artofproblemsolving.com/Wiki go to User:Azjps/geogebra */
import graph; size(6.cm); 
real labelscalefactor = 0.5; /* changes label-to-point distance */
pen dps = linewidth(0.7) + fontsize(10); defaultpen(dps); /* default pen style */ 
pen dotstyle = black; /* point style */ 
real xmin = -2.0406513602548384, xmax = 2.8763425939330705, ymin = -0.8938023179626575, ymax = 2.9620173378013193;  /* image dimensions */

 /* draw figures */
draw((0.,1.5)--(-0.5,0.), linewidth(2.)); 
draw((-0.5,0.)--(0.4801898043036785,0.), linewidth(2.)); 
draw((0.4801898043036785,0.)--(0.5976365702811114,1.5), linewidth(2.)); 
draw((0.,1.5)--(0.5389131872923949,0.75), linewidth(2.)); 
draw((-0.5,0.)--(0.5389131872923949,0.75), linewidth(2.)); 
draw((0.,1.5)--(0.5976365702811114,1.5), linewidth(2.)); 
 /* dots and labels */
dot((0.,1.5),dotstyle); 
label("$A$", (-0.1028548153067883,1.5844680932634196), NE * labelscalefactor); 
dot((-0.5,0.),dotstyle); 
label("$B$", (-0.6894666945597694,-0.14900296565269305), NE * labelscalefactor); 
dot((0.4801898043036785,0.),dotstyle); 
label("$C$", (0.5430773663425617,-0.168776399784816), NE * labelscalefactor); 
dot((0.5976365702811114,1.5),linewidth(4.pt) + dotstyle); 
label("$D$", (0.6221711028710535,1.5515123697098814), NE * labelscalefactor); 
dot((0.5389131872923949,0.75),linewidth(4.pt) + dotstyle); 
label("$F$", (0.6221711028710535,0.6880724126071788), NE * labelscalefactor); 
clip((xmin,ymin)--(xmin,ymax)--(xmax,ymax)--(xmax,ymin)--cycle); 
 /* re-scale y/x */
currentpicture = yscale(0.9776638176638175) * currentpicture; 
 /* end of picture */
\end{asy}

\item In $\triangle ABC, AC=BC, \angle B=2\angle C$. Prove: $AC^2=AB^2+AC\cdot AB$.

 \end{enumerate}

 


\end{document}
